\section{Letter to Your Future Self}
\label{sec:Letter to Your Future Self}

Dear George,


First of all - if you can send back a letter with todays lottery numbers, that
would be great. I hear Euromillions pays well.


Secondly, get back to work! You've got a project to finish! Don't worry though,
you've got an awesome \emph{Continuous Integration} setup, to help make sure that all
the code you write from here on out is pure gold. You're welcome.


In all seriousness, I think you should be seriously proud of the work you've
done on your Continuous Integration and testing frameworks - and should \emph{continue
to use them for the remainder of your project}. Just in case you've forgotten:

\begin{itemize}
  \item They've already caught several bugs (and will continue to catch more)!
  Remember the time you
  \href{https://travis-ci.org/FireEater64/gamq/jobs/93403350#L124}{tried to write
  metrics to a read-only channel}? The time you
  \href{https://travis-ci.org/FireEater64/gamq/builds/92762624}{used a range
  operator that didn't work in Go 1.3}? The time you
  \href{https://travis-ci.org/FireEater64/gamq/jobs/95112027#L143}{found} and
  \href{https://github.com/FireEater64/gamq/commit/5ab48972a8c463859d631f37f75752fd1bfe42d0}{fixed}
  that nasty race condition? None of those would have happened without CI.
  \item It's free marks! I know Agile is behind you now
  (thank goodness\footnote{Only joking - I love Agile :)}), but the report is
  looming. Mentioning the awesome workflow you've come with has got to be worth
  something, right?
  \item It's just plain easier. Do you want to go back to the days of compiling
  the latest version of your binary in from of your supervisor - trying to make
  sure you check out the exact Git version that contains the changes you want to
  show? Why not just
  \href{https://github.com/FireEater64/gamq/releases}{download the pre-built binary from GitHub}?
  Want to make sure your tests are covering everything you need? Why use those
  cryptic command line tools - just check \href{https://coveralls.io/github/FireEater64/gamq?branch=master}{Coveralls}!
\end{itemize}

So - hopefully that's got you (at least somewhat) convinced CI is for life, not
just for Christmas. Travis is already doing a lot of awesome stuff for you, but
here's a few suggestions of ways to take it to the next level:

\begin{itemize}
  \item Automatically collect and plot the results of your running benchmarks\footnote{Currently a manual process}.
  This could be done with some minor adjustments to the build script, and results
  can be sent to some kind of StatsD instance for plotting
  (\href{https://github.com/FireEater64/grafana-statsd-influxdb-docker}{Which you
  were planning on using as part of your project anyway}
  \item Following on from the previous point, you should automatically fail the
  build if any benchmark results exhibits more than, say, 10\% regression compared
  to the previous build. Again - this could be scripted.
  \item 100\% code coverage (or close enough)!
  \item Build OS X and Windows binaries as part of the build. Travis has support
  for building on OS X systems, and you could use something like
  \href{http://www.appveyor.com/}{AppVeyor} to build on Windows. Alternatively,
  \href{http://dave.cheney.net/2015/08/22/cross-compilation-with-go-1-5}{Go
  has excellent support for cross compilation}, you could simply tweak the build
  script to build (and archive) a Windows, OS X and Linux
  binary\footnote{You could even build versions for both ARM and Intel architectures if you get bored} for
  people to download.
\end{itemize}
