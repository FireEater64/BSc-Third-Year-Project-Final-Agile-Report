\section{Letter to Suzanne}
\label{sec:Letter to Suzanne}

Dear Suzanne,


Firstly, I'd like to thank you for running one of the most enjoyable,
industry relevant courses I've taken at Manchester. This isn't just empty
praise to try and score a few marks\footnote{Although if you want to, feel free :)}
- having spoken to a number of friends at other universities, I believe that
Software Engineering is, more than any other subject, one of the worst taught
modules at most universities. Condensing what is (for some people) years of
industry learning into a 12 week course is a daunting challange - and one I feel
has been well met by yourself and the rest of the COMP337 team. With that said,
here's my list of comments/suggestions:

\begin{itemize}
  \item[\textcolor{green}{$\bullet$}] Starting with week 1 - I think the Agile
  lego game was one of the best introductions to Agile practices I've come
  across - I know more than a few colleagues from my placement last year who
  could have learned a lot from it.
  \item[\textcolor{green}{$\bullet$}] I \emph{loved} the fact that the module
  included hands on sessions with writing unit tests/playing with FitNess (although
  using something slightly more modern may have been more useful - I know you're working
  on this).
  \item[\textcolor{green}{$\bullet$}] The lectures throught the semester were
  wonderfully interactive - I think running them in IT407 (rather than a lecture
  threater) really helps with this. I really loved how you merged course content
  with stories from industry year students (sorry for my contributed horror stories).
  \item[\textcolor{red}{$\bullet$}] Whilst I enjoyed our time with FitNess - I
  (personally) feel that we never had quite enough time in the lecture slot to
  explore it fully. Have you ever thought about setting some kind of testing-based
  coursework during the semester? There's a number of different ways something
  like that could be run - but one idea could be to group students together (3-4
  to a group), give each group a bunch of broken code - and get them to first
  write a suite of failing tests, then make the tests pass. You could even have
  a prize\footnote{Chocolate-based of course} for the team with the highest
  code coverage/most useful tests (which almost certainly be the same as the team
  with the highest code coverage). Alternatively, you could pair groups up, and
  have one team responsible for writing tests - whilst the other writes the
  implementation. The testing team get a point (read: chocolate) for each failing
  test they write, the implementation team for each test they pass. Just a thought.
  \item[\textcolor{red}{$\bullet$}] Whilst the material and lab exercises on
  FitNess were awesome - there wasn't much mentioned on how a tool like this
  would fit in a development workflow. Obviously, given the subject of my report,
  I'm biased in this area - but I think a brief mention/demonstration on one
  possible way this could fit into a development flow (check in code - build
  software - run unit tests - run integration tests - run FitNess tests -
  collect code coverage results - archive built binaries) using any popular CI
  tool (Jenkins/Travis/TeamCity/Bamboo etc.) would have been useful. It's a fine
  line between making a course on agile theory, vs just teaching various agile tools,
  but even a generic overview of what makes a good, agile build process, the
  differences, features and tradeoffs associated with Continuous Integration,
  Continuous Deployment etc. would have been good to see.
\end{itemize}

Hopefully there's something interesting in there - though I'm sure you've heard
a lot of it before. Have a great Christmas, and a Happy New Year :)


Thanks,


George
