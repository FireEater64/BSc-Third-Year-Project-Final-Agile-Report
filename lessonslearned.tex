\section{Lessons from Applying the Selected Practice}
\label{sec:Lessons from Applying the Selected Practice}

Despite my initial plan being to set up and host most of my Continuous
Integration infrastructure myself, it became clear after an initial trial
that it would be far easier for me to ensure build repeatability/speed if I used
a cloud-based CI service (some of the reasons for which are detailed below).
With this in mind, I elected to use the popular and
excellent \href{https://travis-ci.org/}{Travis CI} - which provides a free tier
for open-source projects such as mine. Setup was as simple as adding a
`.travis.yml` file to my project, and enabling my GitHub repo for builds on
the Travis.ci admin panel. My initial .travis.yml file can be seen in
Listing~\ref{lst:initialTravis}, and the latest version is available
\href{https://github.com/FireEater64/gamq/blob/master/.travis.yml}{on GitHub}.
There are a multitude of configurable options available, but
my initial, basic configuration in Listing~\ref{lst:initialTravis} consists of
only three:

\begin{description}
  \item[Language] Defines the language of the project being built - in this case
  \href{https://golang.org/}{GoLang}. TravisCI builds (usually) take place
  inside \href{https://www.docker.com/what-docker}{Docker containers}, with the
  'language' section of config dictating which pre-built container is used for
  this particular build. In this case, a language of 'go' will ensure that the
  build container contains all of the binaries and environment variables required
  for building and running go projects.
  \item[Go] This section defines different versions of the go compiler you wish
  to build your product on. When code is checked in - Travis will spin up a
  seperate container for each version specified, and run your complete test suite
  inside each container in parallel. This feature is known as the
  '\href{https://docs.travis-ci.com/user/customizing-the-build/#Build-Matrix}{build matrix}',
  and is both \emph{incredibly} useful for ensuring software consistency on multiple
  different compilers/runtimes (especially useful for interpreted languages such
  as Java), and something which would be hard to replicate outside of a
  containerised build environment (i.e. if I'd chosen to self-host my CI).
  \item[Script] Any custom commands you wish to be run as part of the build.
  Travis contains (pun intended) a standard build script for most languages (which
  are selected via the 'language' configuration detailed above), however builds
  invariably require additional scripts/commands to be run as well - some example
  use cases could be to run additional tests, or deploy build binaries to an
  artifact repository. In this case, the commands I specified execute both my
  unit/integration test suite, and my benchmark suite (Detailed below).
\end{description}

More information on the contents of .travis.yml files can be found in their
\href{https://docs.travis-ci.com/}{excellent documentation}.

\begin{listing}[ht]
\begin{minted}{YAML}
language: go

go:
  - 1.3
  - 1.4
  - tip

script: go test -v ./... -bench=.
\end{minted}
\caption{Initial .travis.yml}
\label{lst:initialTravis}
\end{listing}
