\section{Executive Summary}
\label{sec:Executive Summary}

Selected practice: \textbf{Continuous Integration}.\\
Will continue to use: \textbf{Yes}.

As per my previous report: My project centres around writing an adaptive
application messaging broker, similar in basic function to open source message
brokers like RabbitMQ and ActiveMQ. This broker will be written in the Go
programming language, with a focus on building a high-performance solution
capable of operating in a variety of different network environments - by
tweaking message properties in response to suboptimal network conditions.
For example - in the event that messages are sent over a network which
experiences a high rate of packet loss - the body of the message could be
compressed, to reduce the number of packets needed to contain it and, thereby
reducing the risk of any one packet being lost and requiring retransmission.

The project scope has remained constant since my previous report, with a
slightly increased emphasis on building a functionally correct and performant
broker. The '\emph{autonomic/adaptive}' side of the broker will still be
implemented - and feature in my final project report but, from a software
development/architecture perspective, I find the challange of building a
performant message broker to be slightly more interesting. This slight change of
course actually increases the importance of my Continuous Integration framework,
in order to keep tabs on broker correctness/performance as features are added/code
changes.
